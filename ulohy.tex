\documentclass[12pt,a4paper]{article}
\usepackage[utf8]{inputenc}
\usepackage[czech]{babel}
\usepackage[T1]{fontenc}
\usepackage{amsmath}
\usepackage{amsfonts}
\usepackage{amssymb}
\usepackage{makeidx}
\usepackage{graphicx}
\usepackage[left=2cm,right=2cm,top=2cm,bottom=2cm]{geometry}
\author{Radim Lipka (xlipka02), Roman Ondráček (xondra58), Pavel Raur (xraurp00), David Reinhart (xreinh00)}
\title{Domácí úlohy do Diskrétní matematiky}
\begin{document}
% Vynechání číslování
\pagestyle{empty}

\begin{center}

\includegraphics[height = 96pt]{img/FIT_barevne_CMYK_CZ.pdf} \\

\begin{LARGE}
\textbf{Vysoké učení technické v Brně} \\
\end{LARGE}

\begin{large}
Fakulta informačních technoligií \\
Diskrétní matematika \\
2018~/~2019 \\
\end{large}

\vspace{128pt}

\begin{huge}
\textbf{Sada č.~1} \\
\end{huge}

\vspace{128pt}

\begin{large}
Radim Lipka (xlipka02) \\
Roman Ondráček (xondra58) \\
Pavel Raur (xraurp00) \\
David Reinhart (xreinh00)
\end{large}

\end{center}

\vfill

\newpage

% Povolení číslování
\pagestyle{plain}
\section*{Domácí úloha 1}

\subsection*{Příklad č.~1}

Dokažte nebo vyvraťte protipříkladem následující tvrzení. Pro všechny množiny $X,Y,Z \subseteq U$ platí (doplňky množin uvažujeme vůči množině $U$):

\begin{equation}
\overline{X \cap Y} = \overline{X} \cup \overline{Y} \label{eq:1.1}
\end{equation}

Tvrzení \eqref{eq:1.1} je pravdivé, protože rovnost je dána De Morganovým zákonem. 

\begin{equation}
Y \setminus X = X \setminus Y \label{eq:1.2}
\end{equation}

Tvrzení \eqref{eq:1.2} není pravdivé, protože rozdíl množit není komutatitní.

\textbf{Důkaz:} 
\begin{eqnarray*}
X = \lbrace 1, 2, 3, 4 \rbrace \\
Y = \lbrace 3, 4, 5, 6 \rbrace \\
Y \setminus X \neq X \setminus Y \\
\lbrace 5, 6\rbrace \neq \lbrace 1, 2 \rbrace
\end{eqnarray*}

\begin{equation}
\overline{X \cup Y} \supseteq X \label{eq:1.3}
\end{equation}

\begin{equation}
(X \cup Y) \cap (Y \setminus X) = Y \label{eq:1.4}
\end{equation}

\begin{equation}
X \setminus (Y \cap Z) = (Z \setminus Y) \cup Z \label{eq:1.5}
\end{equation}

\subsection*{Příklad č.~2}

Na množině $X = {1, 2, 3, 4, 5, 6, 7, 8}$ je dána relace $R = \{(x,y)\mid x, y \in X, 3x \text{ dělí } 4y\}$. Zapište relaci $R$ výčtem prvků. Určete její definiční obor a obor hodnot. Nalezněte
relaci $R^{-1}$.

\subsection*{Příklad č.~3}

Uvažujte množinu $\tau_{n}$ všech topologií na množině o $n = 2$ prvcích s uspořádáním
inkluzí. Nakreslete její Hasseův diagram a rozhodněte, zda je $(\tau_{2}, \subseteq)$ svazově uspořádaná.

\subsection*{Příklad č.~4}

Dokažte nebo vyvraťe protipříkladem pro libovolné dvě relace $R_{1}, R_{2}$ na množibě $X$:

\begin{equation}
\rho (R_{1} \cup R_{2}) = \rho (R_{1}) \cup \rho (R_{2})
\end{equation}

\begin{equation}
\sigma (R_{1} \cap R_{2}) = \sigma (R_{1}) \cap \sigma (R_{2})
\end{equation}

\begin{equation}
\tau (R_{1} \cap R_{2}) = \tau (R_{1}) \cap \tau (R_{2})
\end{equation}

\begin{equation}
\sigma(\rho(R_{1})) = \rho(\sigma(R_{1}))
\end{equation}

\begin{equation}
\sigma(\tau(R_{1})) = \tau(\sigma(R_{1}))
\end{equation}

\begin{equation}
\rho(\tau(R_{1})) = \tau(\rho(R_{1}))
\end{equation}

\subsection*{Příklad č.~5}

\section*{Domácí úloha 2}

\subsection*{Příklad č.~6}

Najděte všechny podgrupy $(\mathbb{Z}_{6}, \oplus)$. Které z nich jsou normální? Sestrojte příslušné faktorové grupy.

\subsection*{Příklad č.~7}

Najděte všechny distributivní svazy o 5 prvcích.

\subsection*{Příklad č.~8}

V Booleově algebře $(X, \oplus, \odot, {}^{\prime}, 0, 1)$ zjednodušte výrazy:

\begin{equation}
(a \oplus c) \oplus (c \oplus b) \oplus (b \oplus a) = a \oplus a \oplus b \oplus b \oplus c \oplus c = a \oplus b \oplus c
\end{equation}

\begin{equation}
(x \odot y) \oplus (x \odot z) \oplus (x^{\prime} \odot z^{\prime})^{\prime}
\end{equation}

\begin{equation}
(x^{\prime} \oplus y^{\prime})^{\prime}
\end{equation}

\subsection*{Příklad č.~9}

Buď $G$ úplný graf o $n$ vrcholech. Určete počet cest délky 2, které začínají v pevně zvoleném vrcholu $a$ a končí v jiném pevně zvoleném vrcholu $b$. Kolik existuje takových cest délky 3? Zobecněte tento výsledek pro libovolné $k$, kde $1 \leq\ k < n$.

\subsection*{Příklad č.~10}

Nakreslete graf se šesti vrcholy, který lze nakreslit jedním uzavřeným tahem, ale v němž neexistuje hamiltonovská kružnice. Náležitě zdůvodněte, že tento graf má požadované vlastnosti.

\section*{Domácí úloha 3}

\subsection*{Příklad č.~11}

Je dána soustava lineárních rovnic

\begin{eqnarray*}
x + y - z = -1 \\
x + y -2z = 1 \\
2x + 2y - 2z = 3
\end{eqnarray*}

Každá rovnice určuje rovinu v třírozměrném Eukleidovském prostoru $\mathbb{R}^{3}$. Určete jejich vzájemnou polohu.

\subsection*{Příklad č.~12}

Řešte soustavu lineárních rovnic:

\begin{eqnarray*}
x + 2y + z - w = 6 \\
-x + 3y -z = 8 \\
3x -y + 2w = -1
\end{eqnarray*}

\subsection*{Příklad č.~13}

Jakoukoliv metodou nebo více metodami spočítejte determinant matice:

\begin{equation}
A = 
\begin{pmatrix}
3 & -2 & 0 & 2 \\
-2 & 3 & 3 & -2 \\
0 & 3 & 2 & 1 \\
2 & -2 & 1 & -1
\end{pmatrix}
\end{equation}

\subsection*{Příklad č.~14}

Ve vektorovém prostoru $\mathbb{R}^{4}$ jsou dány vektory $\overline{a}_{1} = 
\begin{pmatrix}
1 \\
1 \\
0 \\
0
\end{pmatrix}$, 
$\overline{a}_{2} = 
\begin{pmatrix}
0 \\
1 \\
1 \\
0
\end{pmatrix}$, 
$\overline{a}_{3} =
\begin{pmatrix}
0 \\
0 \\
1 \\
1
\end{pmatrix}$ a 
$\overline{b}_{1} = 
\begin{pmatrix}
1 \\
0 \\
1 \\
0
\end{pmatrix}$, 
$\overline{b}_{2} =
\begin{pmatrix}
0 \\
2 \\
1 \\
1
\end{pmatrix}$, 
$\overline{b}_{3} =
\begin{pmatrix}
1 \\
2 \\
1 \\
2
\end{pmatrix}$

\subsection*{Příklad č.~15}

Je dáno lineární zobrazení $l: \mathbb{R}^{3} \rightarrow \mathbb{R}^{3}$ předpisem

\begin{equation*}
l\begin{pmatrix}
x_{1} \\
x_{2} \\
x_{3}
\end{pmatrix} = 
\begin{pmatrix}
2x_{1} - 2x_{0 + 3x_{3}} \\
x_{2} + 2x_{3} \\
-3x_{1} - x_{2}
\end{pmatrix}
\end{equation*}.

Najděte jeho maticovou reprezentaci ve standardní bázi a určete bázi a dimenzi prostorů $\textbf{Ker}~l$ a $\textbf{Img}~l$.

\end{document}
