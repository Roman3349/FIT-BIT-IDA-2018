\section*{Domácí úloha 2}

\subsection*{Příklad č.~6}

Najděte všechny podgrupy $(\mathbb{Z}_{6}, \oplus)$. Které z nich jsou normální? Sestrojte příslušné faktorové grupy.

\subsection*{Příklad č.~7}

Najděte všechny distributivní svazy o 5 prvcích.

\subsection*{Příklad č.~8}

V Booleově algebře $(X, \oplus, \odot, {}^{\prime}, 0, 1)$ zjednodušte výrazy:

\begin{eqnarray*}
(a \oplus c) \oplus (c \oplus b) \oplus (b \oplus a) &=&
a \oplus a \oplus b \oplus b \oplus c \oplus c = \\ &=&
a \oplus b \oplus c
\end{eqnarray*}

\begin{eqnarray*}
(x \odot y) \oplus (x \odot z) \oplus (x^{\prime} \odot z^{\prime})^{\prime} &=&
\end{eqnarray*}

\begin{eqnarray*}
(x^{\prime} \oplus y^{\prime})^{\prime} &=&
\end{eqnarray*}

\subsection*{Příklad č.~9}

Buď $G$ úplný graf o $n$ vrcholech. Určete počet cest délky 2, které začínají v pevně zvoleném vrcholu $a$ a končí v jiném pevně zvoleném vrcholu $b$. Kolik existuje takových cest délky 3? Zobecněte tento výsledek pro libovolné $k$, kde $1 \leq\ k < n$.

\subsection*{Příklad č.~10}

Nakreslete graf se šesti vrcholy, který lze nakreslit jedním uzavřeným tahem, ale v němž neexistuje hamiltonovská kružnice. Náležitě zdůvodněte, že tento graf má požadované vlastnosti.

