\section*{Domácí úloha 3}

\subsection*{Příklad č.~11}

Je dána soustava lineárních rovnic

\begin{eqnarray*}
x + y - z &=& -1 \\
x + y -2z &=& 1 \\
2x + 2y - 2z &=& 3
\end{eqnarray*}

Každá rovnice určuje rovinu v třírozměrném Eukleidovském prostoru $\mathbb{R}^{3}$. Určete jejich vzájemnou polohu.

\subsection*{Příklad č.~12}

Řešte soustavu lineárních rovnic:

\begin{eqnarray*}
x + 2y + z - w &=& 6 \\
-x + 3y -z &=& 8 \\
3x -y + 2w &=& -1
\end{eqnarray*}

\begin{equation*}
\left(
\begin{array}{cccc|c}
-1 & 1 & 2 & 1 & 6 \\
0 & -1 & 3 & -1 & 8 \\
2 & 3 & -1 & 0 & -1
\end{array}
\right) =
\left(
\begin{array}{cccc|c}
1 & -1 & -2 & -1 & -6 \\
0 & -1 & 3 & -1 & 8 \\
0 & 5 & 3 & 2 & 11
\end{array}
\right)
\end{equation*}


\subsection*{Příklad č.~13}

Jakoukoliv metodou nebo více metodami spočítejte determinant matice:

\begin{equation}
A =
\begin{pmatrix}
3 & -2 & 0 & 2 \\
-2 & 3 & 3 & -2 \\
0 & 3 & 2 & 1 \\
2 & -2 & 1 & -1
\end{pmatrix}
\end{equation}

\subsection*{Příklad č.~14}

Ve vektorovém prostoru $\mathbb{R}^{4}$ jsou dány vektory
$\overline{a}_{1} =
\begin{pmatrix}
1 \\
1 \\
0 \\
0
\end{pmatrix}$,
$\overline{a}_{2} =
\begin{pmatrix}
0 \\
1 \\
1 \\
0
\end{pmatrix}$,
$\overline{a}_{3} =
\begin{pmatrix}
0 \\
0 \\
1 \\
1
\end{pmatrix}$ a
$\overline{b}_{1} =
\begin{pmatrix}
1 \\
0 \\
1 \\
0
\end{pmatrix}$,
$\overline{b}_{2} =
\begin{pmatrix}
0 \\
2 \\
1 \\
1
\end{pmatrix}$,
$\overline{b}_{3} =
\begin{pmatrix}
1 \\
2 \\
1 \\
2
\end{pmatrix}$. \\
Označme $L_{1} = \langle \lbrace \overline{a}_{1}, \overline{a}_{2}, \overline{a}_{3} \rbrace \rangle$ a $L_{2} = \langle \lbrace \overline{b}_{1}, \overline{b}_{2}, \overline{b}_{3} \rbrace \rangle$. Najděte báze v prostorech $L_{1}$, $L_{2}$, $L_{1} + L_{2}$ a $L_{1} \cap L_{2}$ a stanovte dimenze těchto prostorů.

\begin{eqnarray*}
(\overline{a}_{1}, \overline{a}_{2}, \overline{a}_{3}) &=& 
\begin{pmatrix}
1 & 0 & 0 \\
1 & 1 & 0 \\
0 & 1 & 1 \\
0 & 0 & 1
\end{pmatrix}
%\sim
%\begin{pmatrix}
%1 & 0 & 0 \\
%0 & 1 & 0 \\
%0 & 1 & 0 \\
%0 & 0 & 1
%\end{pmatrix}
\sim
\begin{pmatrix}
1 & 0 & 0 \\
0 & 1 & 0 \\
0 & 0 & 1 \\
0 & 0 & 0
\end{pmatrix} \\
(\overline{b}_{1}, \overline{b}_{2}, \overline{b}_{3}) &=&
\begin{pmatrix}
1 & 0 & 1 \\
0 & 2 & 2 \\
1 & 1 & 1 \\
0 & 1 & 2
\end{pmatrix}
\sim
\begin{pmatrix}
1 & 0 & 0 \\
0 & 1 & 0 \\
0 & 0 & 1 \\
0 & 0 & 0
\end{pmatrix}
\end{eqnarray*}

\subsection*{Příklad č.~15}

Je dáno lineární zobrazení $l: \mathbb{R}^{3} \rightarrow \mathbb{R}^{3}$ předpisem

\begin{equation*}
l\begin{pmatrix}
x_{1} \\
x_{2} \\
x_{3}
\end{pmatrix} =
\begin{pmatrix}
2x_{1} - 2x_{0} + 3x_{3} \\
x_{2} + 2x_{3} \\
-3x_{1} - x_{2}
\end{pmatrix}
\end{equation*}.

Najděte jeho maticovou reprezentaci ve standardní bázi a určete bázi a dimenzi prostorů $\textbf{Ker}~l$ a $\textbf{Img}~l$.

